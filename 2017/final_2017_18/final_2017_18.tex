\documentclass[12pt]{article}

\usepackage{tikz} % картинки в tikz
\usepackage{microtype} % свешивание пунктуации

\usepackage{array} % для столбцов фиксированной ширины

\usepackage{indentfirst} % отступ в первом параграфе

\usepackage{sectsty} % для центрирования названий частей
\allsectionsfont{\centering}

\usepackage{amsmath, amssymb} % куча стандартных математических плюшек


\usepackage{comment}

\usepackage[top=2cm, left=1.2cm, right=1.2cm, bottom=2cm]{geometry} % размер текста на странице

\usepackage{lastpage} % чтобы узнать номер последней страницы

\usepackage{enumitem} % дополнительные плюшки для списков
%  например \begin{enumerate}[resume] позволяет продолжить нумерацию в новом списке
\usepackage{caption}


\usepackage{fancyhdr} % весёлые колонтитулы
\pagestyle{fancy}
\lhead{Time Series}
\chead{}
\rhead{2018-03-29, final exam}
\lfoot{}
\cfoot{}
\rfoot{\thepage/\pageref{LastPage}}
\renewcommand{\headrulewidth}{0.4pt}
\renewcommand{\footrulewidth}{0.4pt}


\let\P\relax
\DeclareMathOperator{\P}{\mathbb{P}}

\usepackage{todonotes} % для вставки в документ заметок о том, что осталось сделать
% \todo{Здесь надо коэффициенты исправить}
% \missingfigure{Здесь будет Последний день Помпеи}
% \listoftodos --- печатает все поставленные \todo'шки


% более красивые таблицы
\usepackage{booktabs}
% заповеди из докупентации:
% 1. Не используйте вертикальные линни
% 2. Не используйте двойные линии
% 3. Единицы измерения - в шапку таблицы
% 4. Не сокращайте .1 вместо 0.1
% 5. Повторяющееся значение повторяйте, а не говорите "то же"



\usepackage{fontspec}
\usepackage{polyglossia}

\setmainlanguage{russian}
\setotherlanguages{english}

% download "Linux Libertine" fonts:
% http://www.linuxlibertine.org/index.php?id=91&L=1
\setmainfont{Linux Libertine O} % or Helvetica, Arial, Cambria
% why do we need \newfontfamily:
% http://tex.stackexchange.com/questions/91507/
\newfontfamily{\cyrillicfonttt}{Linux Libertine O}

\AddEnumerateCounter{\asbuk}{\russian@alph}{щ} % для списков с русскими буквами
%\setlist[enumerate, 2]{label=\asbuk*),ref=\asbuk*}

%% эконометрические сокращения
\DeclareMathOperator{\Cov}{Cov}
\DeclareMathOperator{\Corr}{Corr}
\DeclareMathOperator{\Var}{Var}
\DeclareMathOperator{\E}{E}
\def \hb{\hat{\beta}}
\def \hs{\hat{\sigma}}
\def \htheta{\hat{\theta}}
\def \s{\sigma}
\def \hy{\hat{y}}
\def \hY{\hat{Y}}
\def \v1{\vec{1}}
\def \e{\varepsilon}
\def \he{\hat{\e}}
\def \z{z}
\def \hVar{\widehat{\Var}}
\def \hCorr{\widehat{\Corr}}
\def \hCov{\widehat{\Cov}}
\def \cN{\mathcal{N}}


\begin{document}

\begin{enumerate}


  \item Random variables $y_t$ are independent and normally distributed $\cN(0;1)$. We define random variables $z_t = y_t \cdot y_{t-1}$.
    \begin{enumerate}
      \item Are $z_t$ independent?
      \item Find $\gamma_0 =\Var(z_t)$, $\gamma_1 = \Cov(z_t, z_{t-1})$, $\gamma_2 = \Cov(z_t, z_{t-3})$.
      \item Is $z_t$ a white noise process?
    \end{enumerate}


  \item Consider stationary process $y_t = 3 + 0.5y_{t-1} + u_t + u_{t-1}$, where $u_t$ is a white noise with $\Var(u_t)=\sigma^2_u$. 
    Find the following:
    \begin{enumerate}
      \item $\E(y_t)$, $\gamma_0 = \Var(y_t)$, $\gamma_1 = \Cov(y_t, y_{t-1})$, $\gamma_2 = \Cov(y_t, y_{t-2})$;
      \item Find the first two values of autocorrelation function, $\rho_1$, $\rho_2$;
      \item Find the first two values of partial autocorrelation function, $\phi_{11}$, $\phi_{22}$;
    \end{enumerate}
 


  \item Consider the equation $y_t = 3 + 0.5y_{t-1} + u_t + u_{t-1}$. 
    \begin{enumerate}
      \item If it is possible to express $y_t$ in terms of past $u_{t-i}$ then calculate coefficients before $u_{t-1}$, $u_{t-2}$ and $u_{t-3}$.

      \item If it is possible to express $u_t$ in terms of past $y_{t-i}$ then calculate coefficients before $y_{t-1}$, $y_{t-2}$ and $y_{t-3}$.
    \end{enumerate}


   \item This is the output of seasonal ARIMA model estimation in R for Russian population income.      
     Quarterly data from 1992Q4 to 2015Q4 are used. 
     \begin{verbatim}
Series: y 
ARIMA(0,1,1)(0,1,0)[4] 

Coefficients:
          ma1
      -0.6611
s.e.   0.0811
     \end{verbatim}
     
     \begin{enumerate}
       \item Write down the estimated equation. 
       \item Is the series of population income stationary?
     \end{enumerate}


  \item It is known that $u_{100} = 0.5$, $y_{100}=4.5$, $\Var(u_t)=9$ 
    and $y_t$ is defined by equation $y_t = 3 + 0.5 y_{t-1} + u_t + u_{t-1}$, where $u_t$ is a white noise.

    \begin{enumerate}
      \item Make one-step and two-steps point forecasts: find $\E(y_{101}|\mathcal{F}_{100})$ and $\E(y_{102}|\mathcal{F}_{100})$.
      \item Assuming normal distribution of $u_t$ construct 95\% prediction intervals for $y_{101}$ and $y_{102}$.
    \end{enumerate}



  \item Processes $u_t$ and $v_t$ are independent white noises. 
    The process $y_t$ is defined by equation
    \[
       \begin{cases}
	 y_0 = 0; \\
	 y_t = y_{t-1} + u_t. \\
       \end{cases}
     \]
    The process $z_t$ is defined by equation $z_t = 7 + 0.5z_{t-1} + 2y_{t-1}   -4y_t + v_t$;

    \begin{enumerate}
      \item Find the order of integration of $y_t$ and $z_t$;
      \item Are $y_t$ and $z_t$ cointegrated?
    \end{enumerate}

\end{enumerate}

\end{document}
